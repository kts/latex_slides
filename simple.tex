%%%%%%%%%%%%%%%%%%%%%%%%%%%%%%%%%%%%%%%%%%%%%%
%
% A simple template for LaTeX slides.
%
% With the given Makefile as the same 
% directory as this, run 'make all'
% and see the files created.
% 
% Copyright (c) 2009 Ken Schutte
% Updates, documation, and more at:
% http://kenschutte.com/latex/slides
%
%%%%%%%%%%%%%%%%%%%%%%%%%%%%%%%%%%%%%%%%%%%%%%

\def\SeminarPaperVersion{y} % collapses overlays

\documentclass{seminar}

% for the \inverseslidemag for pstricks figs:
%??
%\input{seminar.bug}

% This for broken installs that flip page wrong way
%  (dvips makes upside down .ps...)
\special{! TeXDict begin /landplus90{true}store end }

% for slideheading:
\usepackage{slidesec}

\usepackage{epsfig}
\usepackage{fancyhdr}
\usepackage{pstricks}
\usepackage{amsmath}

\usepackage{semcolor}

\usepackage{thumbpdf}  % enable thumbnail view in acroread

\usepackage{semlayer}  % slide overlays

\usepackage[ps2pdf,a4paper,
            pdfpagemode=UseNone,urlcolor=NavyBlue]{hyperref}
  
% Workaround to use hyperref and landscape slides
% Remove if you are using portrait slides
% from http://astronomy.sussex.ac.uk/~eddie/soft/tutorial.html
\makeatletter
%    \def\special@paper{297mm,210mm}
  \def\special@paper{280mm,220mm}
\makeatother

\usepackage{graphicx}


% the following makes overlays cumulative:
\makeatletter
\def\pst@initoverlay#1{%
\pst@Verb{%
/BeginOL {dup (all) eq exch TheOL le or {IfVisible not {Visible
/IfVisible true def} if} {IfVisible {Invisible /IfVisible false def} if}
ifelse} def
\tx@InitOL /TheOL (#1) def}}
\makeatother


%%
%% Some handy macros for writing slides
%%
\newcommand{\bi}{\begin{itemize}}
\newcommand{\ei}{\end{itemize}}

\newcommand{\ol}[2]{
\begin{overlay}{#1}{
#2
}\end{overlay}
}


%%%%%%%%%%%%%%%%%%%%%%%%%%%%%

\newcommand{\name}{Your Name}
\newcommand{\talktitle}{The Lecture Title}

%% You might replace the
%% date here with '\today'
%% But, often when you run 
%% latex may not be when you 
%% give the talk...
\newcommand{\talkdate}{January 1, 2000}


%%%%%%%%%%%%%%%%%%%%%%%%%%%%%


\begin{document}


% Don't change/remove the following line!
% It is automatically changed by Makefile
% to create both slides and handouts.
\overlaystrue


% -----------------------------------------------------------------
\begin{slide}

%%\thispagestyle{empty} % no page number of first page...

%% No page number or header/footer on title page:
\pagestyle{empty}

\begin{center}

{\huge \talktitle}

\bigskip

  \name

  Your Organization

  \talkdate

\end{center}

\end{slide}


% -----------------------------------------------------------------

% turn off borders:
\slideframe{none}
\pagestyle{fancy}

%\setlength{\headrulewidth}[0.15pt]
%\setlength{\footrulewidth}[0.15pt]
%\lfoot{\scriptsize \rm \theslide}

\fancyhf{}  % clear all fields
\fancyhead[L]{\scriptsize \sl \talktitle}
\fancyhead[R]{\scriptsize \sl \name $\,$ -- \talkdate}

%\centerslidesfalse \slideframe{none}


%%
%% The slide heading
%% - modify to change text size for slide
%%   titles, etc.
%%
\renewcommand{\makeslideheading}[1]{%
  \begin{center}
  \huge{ \textbf{#1} }
  \end{center}
%  \vspace{.1cm}
}

\centerslidesfalse

% -----------------------------------------------------------------
\begin{slide}
\slideheading{A very simple slide}

Just write stuff here.

Of course, you can add any \LaTeX,

$$
\mathcal{F} \{x[n]\} =
X(\omega) = \sum_{n=-\infty}^{\infty} x[n] e^{-j \omega n}
$$

\end{slide}


% -----------------------------------------------------------------
\begin{slide}
\slideheading{A more typical slide}

  \bi
    \item Often want to use bullets
    \item See the .tex file for handy bi,ei macros I use.
    \bi
      \item As is, this first group is displayed originally,
      \item (all at once)
    \ei

    \ol{1}{
      \item This second group is revealed later
      \bi
        \item Of course you can do it however you want
        \item Everything is 
        \item \textbf{customizable!}
      \ei
    }
  \ei

\end{slide}


\end{document}
